\documentclass[uplatex]{jsarticle}
\usepackage[top=30mm, bottom=25mm, left=30mm, right=30mm]{geometry}
\usepackage{color}
\usepackage[dvipdfmx]{hyperref}
\usepackage{pxjahyper}
 
\special{pdf: minorversion=4}
\AtBeginDvi{\special{pdf:mapfile uptex-ipa.map}}

\setcounter{page}{1}

\title{角ふじ引き継ぎ資料}
\date{}

\begin{document}
\maketitle

\begin{flushright}
 筑波大学管弦楽団\\
 39期技術委員長
\end{flushright}

 \section*{はじめに}

 我々筑波大学管弦楽団の活動拠点である茨城県つくば市は,ラーメン激戦区としてしばしば取り上げられるほど,恵まれた環境にある.
 ともなれば,筑波大生にとってラーメンが大学生活を語るのに欠かせないものとなるのは必然である.
 中でもつくば市上横場にある角ふじは,伝統的に当団の多くの団員の食生活の基盤を成しているといっても過言ではない.

 本資料は,この角ふじに関する情報を共有し,正しい通い方を解説することを目的とする.

 \section{基本データ}

 \begin{tabular}{|c|l|}
  \hline
  店名         & 本家 明神 角ふじ \\
  \hline
  住所         & 〒305-0854 茨城県つくば市上横場 2192-1 \\
  \hline
  TEL          & 029-836-7797 \\
  \hline
  営業日時     & \begin{tabular}{l}
                  ・日月火 11:00-15:00, 18:00-24:00 (L.O. 23:30)\\
                  ・水木金土 11:00-15:00, 18:00-27:00 (L.O. 26:30)\\
                  ※営業日時は頻繁に変わるので留意してください.
                 \end{tabular}\\
  \hline
  駐車場       & 店舗前約4台 + 大勝軒及びうおまつにも駐車可 \\
  \hline
  店内概要     & \begin{tabular}{l}
                  ・カウンター:8席\\
                  ・テーブル:座敷(頑張って7人掛け)×3席
                 \end{tabular}\\
  \hline
  公式ブログ   & \url{http://katimusi.tsukuba.ch} ※2010年より更新がない \\
  \hline
  公式 Twitter & \url{https://twitter.com/kathimushi} ※2014年より更新がない \\
  \hline
  公式 HP      & いつの間にか\href{http://www.katimusi.com/}{眉毛美容液のことが分かるブログ}にドメインを取られていた.\\
  \hline
 \end{tabular}


 \section{主力メニュー}

  \subsection*{・ふじ麺(醤油・味噌) 各750円}
  角ふじの顔.一度食べたらその高い攻撃力の虜になる.
  異様なまでにコシの強い麺とそれを真っ向から迎え撃つ製法不明の強烈ニンニクスープの奏でる交響曲は,常につくばラーメン界にこだましている.
  オプション(コール)として
  \begin{itemize}
   \item ヤサイ(+30円)
   \item アブラ
  \end{itemize}
  を,増し・普通・少なめ・抜きの4つの中から選択できる.
  加えて,テーブル備え付けの「漬けニンニク」を乗せることができる.
  また,通常は太麺だが,細麺の選択も可能だ(筆者は太麺を推奨する).
  各オプションは食券を店員に渡す際に宣言する.
  
  時間帯に依り作る店員が変化し,野菜を過剰に増してくれることがある.
  その際の野菜タワーは,まさしく藝術と呼ぶに相応しい.
  慣れてきたら別途有料オプションの小豚(200円)やバカ豚(380円)も食べてみてほしい.
  特に後者はチャーシューが10枚(最近は8枚くらいかもしれない)になり,その幸福感は格別である.

  \subsection*{・まぜそば(醤油・ごま味噌) 各750円}
  所謂まぜそば.
  麺はふじ麺と同様強靭なもの.
  醤油は,500g を食した際(現在は 400g までしか選べない)に玉ねぎ及びニンニクの辛さに苦しめられた経験がある.
  ごま味噌は辛さを4段階から選択できるが,1辛でも十分に辛いため,注文の際は量に注意せよ.

  オプションでは野菜増しは選べない(かつて一度だけ無理をいって増してもらったことがある)が,代わりにマヨネーズを3倍にできるデブには優しいサービス内容となっている.

  \subsection*{・魂麺 800円}
  筆者は食べたことがないのでよく知らないが,スープの脂分と辛さの相性がいいらしい.
  オプションはふじ麺のものに加え「ジャン増し」を選択できるが,ニンニクとの相乗効果でかなり辛くなってしまうので注意.

  \subsection*{・その他}
  一応つけ麺とかもあります.

  \subsection*{※注意点・補足※}
  \begin{itemize}
   \item 麺の量はデフォルトで 200g であり,そこから 100g 刻みで指定できる.
   \item 「大関」や「横綱」などという名目の専用食券があるので,まずは店頭で確認すべし.
   \item 初見では特にコールはせず,「ふじ麺 400g に漬けニンニクを適量乗せる」のが一般的である.
   \item 500g 以上を注文すると,器がワンランク大きくなり,後述の天地返し等も容易である.ごく稀に 400g でも大きい器を使用することがある.
   \item 角ふじの翌日は便通がよくなりがちである.人によっては腹を下すレベルにまで陥ることがある.
   \item 帰りの車はかなりニンニク臭くなるため,何らかの対策を講じておくとよい.
   \item 基本的にニンニクの残響時間が長いため,当日から翌日の予定に配慮したコールを心がけよ.
   \item 稀に若い店員が作っているケースがあるが,ふじ麺の極意へ到達しているとは言えず,推奨しない.
  \end{itemize}

  \newpage

 \section{角ふじに行くにあたって}
 ここまでで,角ふじの基本的な情報は把握していただけたことだろう.
 次に,より実戦的な内容を述べる.

  \subsection*{・交通手段}
  まず,角ふじに行く際に問題となるのが,交通手段である.
  大学からおよそ 9km の距離にあり,自転車で行くには少々遠い.
  そこで,筆者は自動車で行くことを推奨する.
  もちろん,自転車で行くことも,それはそれで一興である.
  大学からおよそ40分である.
  一時期 ''大学から自転車で25分以内に角ふじに着いたら味玉をサービスしてもらえる'' という非常に胡散臭い噂が出回った(デマだけに?)が,結局真偽は不明である.
  また,最寄りの TX みどりの駅より,店舗の近くまでバスで行くことも可能だ.

  \subsection*{・食べ方}
  結論から言うと,自由に食べていただいて構わない.
  筆者のお勧めは,やはり「天地返し」である.
  分からない方はググっていただくといいだろう.
  400g の天地返しができるようになってくると楽しい.
  麺から一気に貪ることで,空腹をより高位の刺激で満たすことができると確信している.
  \textcolor{red}{\underline{麺は生き物である}}.

  一つだけ忠告しておくが,残すのは厳禁である.
  何gまでを完食できるかというのは古今東西男のロマンだが,残してしまうようでは意味がない.
  自分の実力や体調ときちんと相談し,食べる量を決められよ.
  どうしても苦しい時は,麦茶を汲みにいくという休憩手段があるので,使うとよい.
  しかしこの麦茶がまたクセモノであり,非常に薄いことがある.
  2017年頃に新たな冷水機を導入する以前は,よくカビのような浮遊物を観測していた.

  ここで当団39期クラリネットの団員が考案した「うんのセッティング」(略して「うんセティ」)を紹介したい.
  「うんのセッティング」では麺を 200g に抑え,その代わりにチャーシュー丼を合わせて注文する.
  このチャーシュー丼にふじ麺のスープを垂らしながら食べるのである.
  通常は完全に残してしまうスープをできる限り味わうことのできる,玄人向けのセッティングといえる.
  角ふじに通い慣れた者は試してみるとよい.


  \subsection*{・帰り}
  帰り道にこそ,まさしく我が筑波大学管弦楽団の伝統と言うべき秘密が隠されている.
  それは,マクドナルドつくば学園店(23時をすぎる場合は上横場店)に寄り,口直しをすることである.

  一般的な口直しはマックシェイクによるものである.
  ふじ麺により満たされた胃袋にマックシェイクが浸透する快感は一生忘れることのできないものとなるだろう.
  ふじ麺 400g ヤサイ増しと通常120円のマックシェイクSサイズを合わせるときっかり 1,000 円になり,これを1,000円コースと勝手に呼称している.
  また,角ふじで物足りなかったものはセットメニューを注文するといいだろう.

  筆者は角ふじ後のマックシェイクを欠かしたことがない.
  時々期間限定のマックシェイクが売られているが,様々なマックシェイクにより角ふじの余韻の楽しみ方に変化をつけるのも,また一興であろう.


 \section*{あとがき}
 以上が角ふじの正しい通い方である.
 どうだろうか.
 おそらく,まだ実感が湧かないことだろう.
 角ふじの本当の魅力を感じるには,角ふじに行き,実際に食す意外に手段はないのである.
 読者は直ちに友人を誘い,或いは一人でも角ふじに行くべきである.

  \subsection*{本資料の更新に関して}
  角ふじは日々進化し続ける.
  そのため,本資料もそれに応じてアップデートされる必要がある.

  本資料は \url{https://github.com/23prime/kdfj} にて公開されており,元の LaTeX ファイルも同リポジトリにある.
  資料の更新の際には,なるべく GitHub の Pull request 機能を用いてほしい.

  \vspace{6mm}
 最後に,本資料を読んでいる諸君らには,団の誇り高き伝統をさらに次世代へ伝承していくことを約束してほしい.

 以上をもって,筆者からの引き継ぎとしたい.

 \newpage

 \vspace*{\stretch{1}}
 \begin{center}
  {\bf {\huge {\textcolor{red}{
  こんなものを最後まで読んでいる暇があったら,\\楽譜でも見ながら合奏音源聞いてください.
  }}}}
 \end{center}
 \vspace{\stretch{2}}

\end{document}